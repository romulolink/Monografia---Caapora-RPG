\chapter{Introdução}
\label{cap:introducao}
 
 
Em 1958 surgia o primeiro jogo eletrônico com o nome de "Tênis para dois", criado pelo físico William Higinbotham.
O jogo era processado por um computador analógico e exibido em um osciloscópio.\\
Em 1961 um grupo de estudantes do MIT desenvolveram o jogo Spacewar!. No jogo o objetivo era que os jogadores controlassem suas naves num ambiente escuro tentando abater o adversário. O jogo era desenvolvido num grande computador, que ocupava o espaço de uma sala inteira, o DEC PDP-1, este computador tinha como objetivo executar cálculos em geral porem apos o jogo Spacewar! o computador ficou famoso pela possibilidade de entreter outros.\\
Foi então em 1972 que o primeiro jogo comercial foi lançado pela Atari, o Pong. Era um jogo simples e intuitivo, características que o fez ser tao popular para a época. (CLUA e BITTENCOURT, 2005) 

\begin{figure}[h!]
		\centering
		\Caption{\label{fig:exemplo-1}Jogo 'Pong' criado pel Atari em 1972}	
		\UECEfig{}{
			\fbox{\includegraphics[width=7cm]{figuras/193px-Pong}}
		}{
			\Fonte{Wikipedia, 2015}			
		}	
	\end{figure}


Este citados foram os responsáveis pela grande diversidade e popularidade que os jogos alcançam ultimamente.

Atualmente devido a popularização da internet e dos dispositivos moveis, em destaque o uso de smartphones no qual evidenciado na pesquisa realizada pela consultoria Nielsen 68 milhões de brasileiros utilizam smartphones (NIELSEN,2015).Tendo em base a estatística da Nielsen, as palavras do diretor geral da Pontomobi, Renato Virgilli, tornam-se ainda mais expressivas "Com o número cada vez maior de celulares, de smartphones e de tablets nas mãos dos usuários, cresce o interesse por aplicativos, seja para diversão ou entretenimento" (VIRGILLI)
E que de acordo com a empresa Flurry do grupo Yahoo, o uso de aplicativos cresceu em media 76\% em 2014, conforme demostrado no gráfico abaixo.\begin{figure}[h!]
		\centering
		\Caption{\label{fig:exemplo-2}Crescimento uso de aplicativos em 2014}	
		\UECEfig{}{
			\fbox{\includegraphics[width=10cm]{figuras/flurry76}}
		}{
			\Fonte{Flurry, 2015}			
		}	
	\end{figure}
	
	
Os aplicativos da categoria jogos tornaram-se um passatempo global, onde pode ser visto na pequisa  realizada pela empresa Flurry que os jogadores globais (plataforma Android) gastam em media 37 minutos por dia jogando. Desses 37 minutos, a categoria que prevalece e a de 'Casuais', conforme pode ser observado no gráfico abaixo.\begin{figure}[h!]
		\centering
		\Caption{\label{fig:exemplo-3}Distribuição de jogos por Categoria}	
		\UECEfig{}{
		\fbox{\includegraphics[width=7cm]{figuras/categoria}}
		}{
			\Fonte{Flurry, 2014}			
		}	
	\end{figure}
	
	
Uma das vias para qual o mercado de aplicativos de jogos tem mostrado desenvolvimento eficiente é como ferramenta pedagógica.
"O aluno é totalmente ativo ao usar um aplicativo, diferente de uma TV, que ele tem uma postura mais passiva. 
Os envolvidos no processo passam de consumidores a produtores de conteúdo, e a ter mais autonomia e criatividade, habilidades que serão demandadas no seu futuro profissional" (TORI)
Considerações essa de Tori, coordenador do Laboratório de Tecnologias Interativas da USP (Universidade de São Paulo).
de que o aplicativo tem a capacidade de auxiliar o desenvolvimento de uma tarefa, e que a escola não pode viver numa realidade desconectada da do aluno.

A professora e diretora-executiva do Instituto Educadigital, Priscila Gonsales vem com o conceito de que recurso digitais
e audiovisuais auxiliam a interação e visualização, aponta também que o aplicativo não substitui a metodologia tradicional e a utilização da ferramenta deva ser contextual ao conteúdo abordado.

Baseando-se nos itens descritos acima: popularidade no uso de aplicativos e que este tem eficiência no uso didático, foi desenvolvido e sera apresentado neste documento o jogo digital Caapora, que tem como objetivo ensinar crianças sobre a importância da preservação da mata e ressaltar a cultura folclórica brasileira.
Aplicativo este desenvolvido para dispositivos moveis.

