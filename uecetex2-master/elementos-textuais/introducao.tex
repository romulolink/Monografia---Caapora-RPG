\chapter{Introdução}
\label{cap:introducao}
O crescimento da internet e da Web teve um impacto significativo em diversas áreas, como negócios, comercio e indústrias, tornando-se aspecto central de muitas aplicações em diferentes áreas (SOUZA, 2015). 
Os principais fatores que contribuíram para tal crescimento foram às percepções das vantagens que a Web fornecia como um meio de negócio e o surgimento de novas ferramentas e tecnologias para o desenvolvimento e emprego de aplicações. 
Junto com a demanda do crescimento de aplicações para Web, os jogos virtuais vêm conquistando um espaço cada vez maior e respeitável.  Segundo uma pesquisa realizada pelo Ibope (IBOPE, xxxx), 23% dos brasileiros são jogadores assíduos ou eventuais. Revelado também numa outra pesquisa realizada pela a empresa de consultoria PWC (Price Waterhouse Coopers) (PWC, xxxx), o mercado de jogos no brasil é o quarto maior do mundo, sendo constatado também que 42% dos brasileiros jogam via computador e 67% via console. 
 Como demostrado na pesquisas realizadas, o brasileiro vem a cada dia se simpatizando mais com os jogos digitais. Portanto jogos digitais tornaram-se um entretenimento para pessoas de diferentes idades e a presença desta modalidade de divertimento, especialmente atrelada ao uso computador. E para finalidade pedagógica é cada vez mais significativa e conveniente. 
Segundo Orso “A criança precisa ser alguém que joga para que mais tarde, saiba ser alguém que age, convivendo sadiamente com as regras do jogo da vida. Saber ganhar e perder deveria acompanhar a todos sempre” (Orso apud Grübel e Bez, 2006).             
Com a disseminação do uso do computador e a fácil inserção destes dispositivos, ganha-se maiores possibilidades no processo de aprendizagem, já mencionados por Oliveira (2001) "O ato de ensinar e aprender ganha novo suporte com o uso de diferentes tipos de softwares educacionais, de pesquisas na Internet e de outras formas de trabalhos com o computador." 
Tendo como base de que um jogo digital pode ter uma finalidade didática, foi desenvolvido, e é apresentado neste documento, o jogo digital Caapora, que tem como objetivo ensinar crianças sobre a importância da preservação da mata Amazônica em um aplicativo desenvolvido para dispositivos móveis. 
