\chapter{Conclusão}
\label{chap:conclusoes-e-trabalhos-futuros}

Este projeto tem o objetivo de conscientizar as crianças sobre o desmatamento que ocorre em todo Brasil, principalmente na Floresta Amazônica. Para atingir esse objetivo, foi desenvolvido um jogo que é uma atividade rica em aprendizagem com o intuito de mostrar o que o país esta enfrentando e o quanto isso é grave.

O jogo possui características importantes, já que o gênero (RPG com elementos de ação), faz com que o jogador assuma o personagem dentro do jogo, dando a ele uma perspectiva do que a natureza esta enfrentando.

Todo o desenvolvimento do jogo foi criado dentro do ambiente da Unity, isso devido à facilidade que o software oferece, como uma grande variedade de componentes que auxiliam na criação do jogo e também devido a ele ser multiplataforma, ou seja, apesar de ele ser para a plataforma Android, o jogo pode ser migrado para uma outra plataforma.

Com relação a performance, o jogo atingiu os requisitos de rodar à 30 quadros por segundo, visto que em jogos 2D onde são usados clipes de animação mais simples a taxa de 12 FPS é a mínima aceitável. 

Por fim, o projeto atendeu as expectativas visto que a principal mensagem que o jogo deveria transmitir foi alcançada, já que ele ilustra bem o que esta ocorrendo na floresta, assim como também foi utilizado as tecnologias e ferramentas atuais.

\section{Trabalhos Futuros}
\label{sec:trabalhos-futuros}

Como trabalhos futuros a serem melhorados e implementados dentro do jogo, estima-se as seguintes funcionalidades:

\begin{alineascomponto}
	
   \item Criação de mais 3 cenários;
   \item Inclusão de mais animais;
   \item Criação do Curupira, que será o segundo personagem que o jogador poderá escolher;
   \item Melhoria do sistema de inteligência artificial utilizado no jogo, visando a realidade dos movimentos executados pelo IA;

	\end{alineascomponto}