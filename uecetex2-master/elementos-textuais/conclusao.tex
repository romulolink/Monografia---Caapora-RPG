\chapter{Conclusão}
\label{chap:conclusoes-e-trabalhos-futuros}

Este projeto foi criado com o objetivo de conscientizar nosso público alvo sobre o desmatamento que anda ocorrendo em todo Brasil, principalmente na Floresta Amazônica. Para atingir esse objetivo, criamos um jogo que é uma atividade rica em aprendizagem com o intuito de mostrar o que o nosso país anda enfrentando e o quanto isso é grave.

O jogo possui características importantes já que o gênero escolhido, no caso o ARPG, faz com que o jogador assuma o personagem dentro do jogo, dando a ele uma perspectiva do que a natureza esta enfrentando.

Todo o desenvolvimento do jogo foi feito dentro do ambiente da Unity, isso devido à facilidade que o software oferece, como uma grande variedade de componentes que auxiliam na criação do jogo e também devido a ele ser multiplataforma, ou seja, apesar de escolhermos a plataforma Android, o jogo pode ser migrado para uma outra plataforma.

Durante o tempo de desenvolvimento tivemos algumas dificuldades visto que não tínhamos conhecimento sobre a ferramenta e nosso tempo para concluir o projeto era curto, no entanto o resultado foi satisfatório, pois alcançamos grande parte do nosso objetivo que é conscientizar nosso publico alvo e utilizar alguns conceitos acadêmicos adquiridos ao longo desses anos.

\section{Trabalhos Futuros}
\label{sec:trabalhos-futuros}

Como trabalhos futuros a serem melhorados e implementados dentro do jogo, estima-se as seguintes funcionalidades:

\begin{alineascomponto}
	
   \item Criação de mais 3 cenários;
   \item Inclusão de mais animais;
   \item Criação do Curupira, que será o segundo personagem que o jogador poderá escolher;
   \item Melhoria do sistema de inteligência artificial utilizado no jogo, visando a realidade dos movimentos executados pelo IA;

	\end{alineascomponto}